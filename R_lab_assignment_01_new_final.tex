% Options for packages loaded elsewhere
\PassOptionsToPackage{unicode}{hyperref}
\PassOptionsToPackage{hyphens}{url}
%
\documentclass[
]{article}
\usepackage{amsmath,amssymb}
\usepackage{iftex}
\ifPDFTeX
  \usepackage[T1]{fontenc}
  \usepackage[utf8]{inputenc}
  \usepackage{textcomp} % provide euro and other symbols
\else % if luatex or xetex
  \usepackage{unicode-math} % this also loads fontspec
  \defaultfontfeatures{Scale=MatchLowercase}
  \defaultfontfeatures[\rmfamily]{Ligatures=TeX,Scale=1}
\fi
\usepackage{lmodern}
\ifPDFTeX\else
  % xetex/luatex font selection
\fi
% Use upquote if available, for straight quotes in verbatim environments
\IfFileExists{upquote.sty}{\usepackage{upquote}}{}
\IfFileExists{microtype.sty}{% use microtype if available
  \usepackage[]{microtype}
  \UseMicrotypeSet[protrusion]{basicmath} % disable protrusion for tt fonts
}{}
\makeatletter
\@ifundefined{KOMAClassName}{% if non-KOMA class
  \IfFileExists{parskip.sty}{%
    \usepackage{parskip}
  }{% else
    \setlength{\parindent}{0pt}
    \setlength{\parskip}{6pt plus 2pt minus 1pt}}
}{% if KOMA class
  \KOMAoptions{parskip=half}}
\makeatother
\usepackage{xcolor}
\usepackage[margin=1in]{geometry}
\usepackage{graphicx}
\makeatletter
\def\maxwidth{\ifdim\Gin@nat@width>\linewidth\linewidth\else\Gin@nat@width\fi}
\def\maxheight{\ifdim\Gin@nat@height>\textheight\textheight\else\Gin@nat@height\fi}
\makeatother
% Scale images if necessary, so that they will not overflow the page
% margins by default, and it is still possible to overwrite the defaults
% using explicit options in \includegraphics[width, height, ...]{}
\setkeys{Gin}{width=\maxwidth,height=\maxheight,keepaspectratio}
% Set default figure placement to htbp
\makeatletter
\def\fps@figure{htbp}
\makeatother
\setlength{\emergencystretch}{3em} % prevent overfull lines
\providecommand{\tightlist}{%
  \setlength{\itemsep}{0pt}\setlength{\parskip}{0pt}}
\setcounter{secnumdepth}{-\maxdimen} % remove section numbering
\ifLuaTeX
  \usepackage{selnolig}  % disable illegal ligatures
\fi
\IfFileExists{bookmark.sty}{\usepackage{bookmark}}{\usepackage{hyperref}}
\IfFileExists{xurl.sty}{\usepackage{xurl}}{} % add URL line breaks if available
\urlstyle{same}
\hypersetup{
  hidelinks,
  pdfcreator={LaTeX via pandoc}}

\author{}
\date{\vspace{-2.5em}}

\begin{document}

\#1. Consider mtcars and iris (Motor Trend Car Road Tests and Iris)
\#data sets available in R-statistical software, and find \#i. Mean,
median, mode, 1st quartile, 2nd quartile, 3rd quartile, variance,
\#standard deviation, and covariance between any two variables using
inbuilt \#functions and by writing your own function with proper
documentation. rm(list=ls(all=TRUE)) \#checklen function used in almost
all programs checklen=function(x)\{ i=1; while(!is.na(x{[}i,1{]}))\{
i=i+1; if(is.na(x{[}i,1{]}))\{ break; \} \} checklen=i-1 checklen \}
\#--------finding mean of column mpg and Sepal.Length x=mtcars y=iris
mean(x{[},1{]}) mean(y{[},1{]}) \#ans=20.09062 for mtcars \#ans=5.843333
for iris

\#for finding sum of observations of mpg column and Sepal.Length \#here
we are passing dataset and column number as parameter findMean =
function(x,n)\{ i=1; \#to keep track of iterations sum=0;
total=checklen(x) for(i in 1:total)\{ sum=sum+x{[}i,n{]} \} \#mean= sum
of observations/total number of observations findMean=sum/checklen(x)
findMean \} \#ans 20.09062 for mtcars \#ans 5.843333 for iris

\#--------finding median of column mpg and Sepal.Length
median(x{[},1{]}) median(y{[},1{]}) \#ans=19.2 for mtcars \#ans=5.8 for
iris findMedian=function(x)\{ \#code below is to sort the array. t is a
temporary variable used in swapping for(i in 1:(checklen(x)-1))\{ for(j
in (i+1):checklen(x))\{ if(x{[}j,1{]}\textless x{[}i,1{]})\{
t=x{[}i,1{]} x{[}i,1{]}=x{[}j,1{]} x{[}j,1{]}=t \} \} \} \#if number of
elements odd then median at (n+1)/2th position \# \%\% used to check
remainder if(checklen(x)\%\%2==1)\{ med=x{[}((checklen(x)+1)/2),1{]} \}
else\{ m1=x{[}((checklen(x))/2),1{]} m2=x{[}(((checklen(x))/2)+1),1{]}
med=(m1+m2)/2 \} findMedian=med findMedian \} \#ans=19.2

\#--------finding mode of column mpg and Sepal.Length f=table(x{[},1{]})
table(y{[},1{]}) \#frequencies of numbers shown. 10.4, 15.2, 19.2, 21,
21.4, 22,8, 30.4 occur \#occur two times \#function to find mode without
in build function findMode=function(x)\{ \#sorting array for(i in
1:(checklen(x)-1))\{ for(j in ((i+1):(checklen(x))))\{
if(x{[}j,1{]}\textless x{[}i,1{]})\{ t=x{[}i,1{]} x{[}i,1{]}=x{[}j,1{]}
x{[}j,1{]}=t \} \} \}

h=rep(0,(checklen(x))) \#checking for frequency and updating for(i in
1:(checklen(x)-1))\{ for(j in (i+1):checklen(x))\{
if(x{[}i,1{]}==x{[}(j),1{]})\{ h{[}i{]}=h{[}i{]}+1 \} \}

\} \#finding max frequency max=h{[}1{]} for(i in 1:(checklen(x)))\{
if(h{[}i{]}\textgreater max)\{ max=h{[}i{]} \} \} \#printing numbers
with max frequency for(i in 1:(checklen(x)))\{ if(h{[}i{]}==max)\{
print(x{[}i,1{]}) \} \} \}

\#--------finding quartiles and median of column mpg and Sepal.Length
quartile=quantile(x{[},1{]},c(0.25,0.5,0.75)) \#this in build function
gives value of 1st quartile, median and 3rd quartile \#ans = 15.425,
19.2, 22.8 for mtcars \#ans = 5.1, 5.8, 6.4 \#creating our own function
findQuartile=function(x,n)\{ stopifnot(n\textgreater= 1 \&\&
n\textless=3) \#sorting array for(i in 1:(checklen(x)-1))\{ for(j in
(i+1):(checklen(x)))\{ if(x{[}j,1{]}\textless x{[}i,1{]})\{ t=x{[}i,1{]}
x{[}i,1{]}=x{[}j,1{]} x{[}j,1{]}=t \} \} \} \#first quartile if(n==1)\{
findQuartile=(x{[}as.integer((checklen(x)+1)/4),1{]}) findQuartile \}
\#third quartile else if(n==3)\{
findQuartile=(x{[}as.integer(3*(checklen(x)+1)/4),1{]}) findQuartile \}
\#median else if(n==2)\{
findQuartile=(x{[}as.integer((checklen(x)+1)/2),1{]}) findQuartile \} \}

\#--------finding variance of column mpg and Sepal.Length var(x{[},1{]})
var(y{[},1{]}) \#in build function, ans=36.3241 \#in build function,
ans=0.685693

findVar=function(x)\{ sum=0 i=1 mean=findMean(x) for(i in
1:checklen(x))\{ sum=sum+((x{[}i,1{]}-mean)\^{}2)/(checklen(x)-1) \}
findVar=sum findVar \}

\#--------finding standard deviation of column mpg and Sepal.Length
sd(x{[},1{]}) sd(y{[},1{]}) \#in build function, ans=6.026948 for mtcars
\#ans=0.8280661 for iris findSd=function(x)\{ findSd=sqrt(findVar(x))
findSd \}

\#--------finding covariance of column mpg and Sepal.Length
cov(x{[},1{]},x{[},2{]}) \#-9.172379 cov(y{[},1{]},y{[},2{]})
\#-0.042434 findCov=function(x)\{ sum=0 for(i in 1:(checklen(x)))\{
sum=sum+(((x{[}i,1{]}-findMean(x,1))*(x{[}i,2{]}-findMean(x,2)))/(checklen(x)-1))
\} findCov=sum findCov \}

\#--------ii. Describe data using summary in R summary(x) summary(y)

\#--------iii. Make use of histogram, bar chart, pie chart and boxplot
to \#illustrate data on different variables.

x=mtcars; \#for(i in 1:11)\{ hist(x{[},1{]}, xlab=``mpg'',
ylab=``frequency'', main=(``mpg of mtcars''), col=``lavender'' ) \#\}
hist(x{[},2{]}, xlab=``cyl'', ylab=``frequency'', main=(``cyl of
mtcars''), col=``lavender'' )

hist(x{[},3{]}, xlab=``disp'', ylab=``frequency'', main=(``disp of
mtcars''), col=``lavender'' )

hist(x{[},4{]}, xlab=``hp'', ylab=``frequency'', main=(``hp of
mtcars''), col=``lavender'' )

hist(x{[},5{]}, xlab=``drat'', ylab=``frequency'', main=(``drat of
mtcars''), col=``lavender'' )

hist(x{[},6{]}, xlab=``wt'', ylab=``frequency'', main=(``wt of
mtcars''), col=``lavender'' )

hist(x{[},7{]}, xlab=``qsec'', ylab=``frequency'', main=(``qsec of
mtcars''), col=``lavender'' )

hist(x{[},8{]}, xlab=``vs'', ylab=``frequency'', main=(``vs of
mtcars''), col=``lavender'' )

hist(x{[},9{]}, xlab=``am'', ylab=``frequency'', main=(``am of
mtcars''), col=``lavender'' )

hist(x{[},10{]}, xlab=``gear'', ylab=``frequency'', main=(``gear of
mtcars''), col=``lavender'' )

hist(x{[},11{]}, xlab=``carb'', ylab=``frequency'', main=(``carb of
mtcars''), col=``lavender'' )

\#pie chart on number of cylinders pie(table(x{[},2{]}), main=``Number
of Cylinders'' )

boxplot(mtcars\$mpg, main=`Distribution of mpg values', ylab=`mpg',
col=`lavender', border=`black')

barplot( table(x{[},2{]}), xlab=``number of cylinders'', ylab=``number
of cars'', main=``cylinders in mtcars'' )

\#hist, pie and bar chart for iris y=iris;

hist(iris\$Sepal.Length, col=`steelblue', main=`Histogram',
xlab=`Length', ylab=`Frequency')

hist(iris\$Sepal.Width, col=`steelblue', main=`Histogram', xlab=`Width',
ylab=`Frequency')

hist(iris\$Petal.Length, col=`steelblue', main=`Histogram',
xlab=`Length', ylab=`Frequency')

hist(iris\$Petal.Width, col=`steelblue', main=`Histogram', xlab=`Width',
ylab=`Frequency')

\#hist for species throws error as it requres numeric inputs \#pie chart
on number of species pie(table(y{[},5{]}), main=``Species of Irises'' )

boxplot(mtcars\$mpg, main=`Distribution of mpg values', ylab=`mpg',
col=`lavender', border=`black')

\#barplot for iris table barplot( table(y{[},5{]}), \#gives frequency of
species xlab=``number of Species'', ylab=``frequency'', main=``frequency
of species'' )

\#--------2. Make use of plot, lines and legend functions in R to plot
the \#graph of PMF/PDFs and CDFs of following statistical distributions
\#corresponding to various parameter values on the same x-axis.

\#i. binomial \#dbinom x = 1:80

\hypertarget{size-80-prob-0.2}{%
\section{size = 80, prob = 0.2}\label{size-80-prob-0.2}}

plot(dbinom(x, size = 80, prob = 0.2), type = ``l'', main = ``Binomial
probability function'', ylab = ``P(X = x)'', xlab = ``Number of
successes'', col=``black'', lty=1, lwd = 3, )

\hypertarget{size-80-prob-0.3}{%
\section{size = 80, prob = 0.3}\label{size-80-prob-0.3}}

lines(dbinom(x, size = 80, prob = 0.3), type = ``l'', lty=2, lwd = 2,
col = ``blue'')

\hypertarget{size-80-prob-0.4}{%
\section{size = 80, prob = 0.4}\label{size-80-prob-0.4}}

lines(dbinom(x, size = 80, prob = 0.4), type = ``l'', lty = 3, lwd= 1,
col = ``red'')

\hypertarget{add-a-legend}{%
\section{Add a legend}\label{add-a-legend}}

legend(``topright'', legend = c(``80 0.2'', ``80 0.3'', ``80 0.4''),
title = ``size prob'', lty=c(1,2,3), lwd=c(3,2,1),
col=c(``black'',``blue'',``red'') ) \#pbinom

\hypertarget{size-80-prob-0.2-1}{%
\section{size = 80, prob = 0.2}\label{size-80-prob-0.2-1}}

plot(pbinom(x, size = 80, prob = 0.2), type = ``l'', lty=1, lwd = 3,
main = ``Binomial distribution function'', xlab = ``Number of
successes'', ylab = ``F(x)'', col=``black'' )

\hypertarget{size-80-prob-0.3-1}{%
\section{size = 80, prob = 0.3}\label{size-80-prob-0.3-1}}

lines(pbinom(x, size = 80, prob = 0.3), type = ``l'', lwd = 2, lty = 2,
col=``red'' )

\hypertarget{size-80-prob-0.4-1}{%
\section{size = 80, prob = 0.4}\label{size-80-prob-0.4-1}}

lines(pbinom(x, size = 80, prob = 0.4), type = ``l'', lty=3, lwd = 1,
col = ``blue'')

\hypertarget{add-a-legend-1}{%
\section{Add a legend}\label{add-a-legend-1}}

legend(``bottomright'', legend = c(``80 0.2'', ``80 0.3'', ``80 0.4''),
title = ``size prob'', lty=c(1,2,3), lwd=c(3,2,1),
col=c(``black'',``red'',``blue'') )

\#ii. poisson x=0:50 \#using dpois to find pmf lambda=5
plot(x,dpois(x,lambda), type=`l', main=``Poisson Probability Mass
Function'', ylab=``P(X=x)'', xlab=(``Number of events''), col=`black',
lty=1, lwd=3 ) lambda=10 lines(dpois(x,lambda), type=`l', col=`red',
lty=2, lwd=2 ) lambda=20 lines(x,dpois(x,lambda), type=`l', col=`green',
lty=3, lwd=1 )

legend(``topright'', legend = c(``5'',``10'',``20''), title=``lambda'',
col=c(``black'',``red'',``green''), lty=c(1,2,3), lwd=c(3,2,1), )
\#ppois lambda=5 plot(x,ppois(x,lambda), type=`l', main=``Poisson CDF'',
ylab=``F(x)'', xlab=(``Number of events''), col=`black', lty=1, lwd=3 )
lambda=10 lines(ppois(x,lambda), type=`l', col=`red', lty=2, lwd=2 )
lambda=20 lines(x,ppois(x,lambda), type=`l', col=`green', lty=3, lwd=1 )

legend(``topright'', legend = c(``5'',``10'',``20''), title=``lambda'',
col=c(``black'',``red'',``green''), lty=c(1,2,3), lwd=c(3,2,1), )

\#iii. Uniform x \textless- seq(-4, 4, length=100) plot(x,
dunif(x,min=-3, max=3), type = `l', lty=1, lwd = 2, ylim = c(0, .3),
col=`blue', xlab=`x', ylab=`Probability', main=`Uniform Distribution
Plot') lines(x,dunif(x, min=-2, max=2), type=`l', col=`green', lty=2,
lwd=1 )

legend(``topright'', legend = c(``-3 to 3'',``-2 to 2''), title=``min
max values'', col=c(``blue'',``green''), lty=c(1,2), lwd=c(2,1), cex=0.6
) \#punif plot(x, punif(x,min=-3, max=3), type = `l', lty=1, lwd = 2,
ylim = c(0, 1.5), col=`blue', xlab=`x', ylab=`Probability',
main=`Uniform Distribution Plot') lines(x,punif(x, min=-2, max=2),
type=`l', col=`green', lty=2, lwd=1 )

legend(``topright'', legend = c(``-3 to 3'',``-2 to 2''), title=``min
max values'', col=c(``blue'',``green''), lty=c(1,2), lwd=c(2,1), cex=0.6
) \#iv. Exponential Function \#dexp x=seq(0,8,0.1) lambda=0.5
plot(x,dexp(x,lambda),typ=``l'', ylab=``P(x)'', xlab=``x'', col=`black',
lty=1, lwd=3 ) lambda=1 lines(x,dexp(x,lambda), type=`l', col=`red',
lty=2, lwd=2 ) lambda=2 lines(x,dexp(x,lambda), type=`l', col=`green',
lty=3, lwd=1 )

legend(``topright'', legend = c(``0.5'',``1'',``2''), title=``lambda'',
col=c(``black'',``red'',``green''), lty=c(1,2,3), lwd=c(3,2,1), )

\#pexp x=seq(0,8,0.1) lambda=0.5 plot(x,pexp(x,lambda),typ=``l'',
ylab=``F(x)'', xlab=``x'', col=`black', lty=1, lwd=3 ) lambda=1
lines(x,pexp(x,lambda), type=`l', col=`red', lty=2, lwd=2 ) lambda=2
lines(x,pexp(x,lambda), type=`l', col=`green', lty=3, lwd=1 )

legend(``topright'', legend = c(``0.5'',``1'',``2''), title=``lambda'',
col=c(``black'',``red'',``green''), lty=c(1,2,3), lwd=c(3,2,1), )

\#v. gamma function \#dgamma x=seq(0,2,0.01) curve(dgamma(x, shape=2,
rate=1), from=0, to=5, ylim=c(0,1), col=`black')

curve(dgamma(x, shape=3, rate=2), from=0, to=7, col=`red', add=TRUE)

curve(dgamma(x, shape=4, rate=3),\\
from=0, to=10, col=`blue', add=TRUE) legend(``topright'',
legend=c(``shape 2, scale 1'',``shape 3, scale 2'',``shape 4, scale
3''), text.col=c(``black'',``red'',``blue''), cex=0.5 \#for setting the
text size ) \#pgamma x=seq(0,2,0.01) curve(pgamma(x, shape=2, rate=1),
from=0, to=5, ylim=c(0,1), col=`black')

curve(pgamma(x, shape=3, rate=2), from=0, to=7, col=`red', add=TRUE)

curve(pgamma(x, shape=4, rate=3),\\
from=0, to=10, col=`blue', add=TRUE) legend(``bottomright'',
legend=c(``shape 2, scale 1'',``shape 3, scale 2'',``shape 4, scale
3''), text.col=c(``black'',``red'',``blue''), cex=0.5 \#for setting the
text size )

\#vi. normal distribution \#dnorm x \textless- seq(-4, 8, 0.1)

\#mean=0, sd=1 plot(x, dnorm(x, mean = 0, sd = 1), type = ``l'', ylim =
c(0, 0.6), xlab=``x'', ylab = ``P(X==x)'', lwd = 2, col = ``red'')
\#mean=3, sd=1 lines(x,dnorm(x, mean=3, sd=1), col=``blue'', lty=1,
lwd=2 ) \#legend legend(``topright'', legend = c(``0 1'', ``3 1''), col
= c(``red'', ``blue''), title = expression(paste(mu,'' ``,sigma)),
title.adj = 0.9, lty = 1, lwd = 2)

\#mean same, sd different \# Mean 1, sd 1 plot(x, dnorm(x, mean = 1, sd
= 1), type = ``l'', ylim = c(0, 1), ylab = ``P(X==x)'', lwd = 2, col =
``red'') \# Mean 1, sd 0.5 lines(x, dnorm(x, mean = 1, sd = 0.5), col =
``blue'', lty = 1, lwd = 2)

\hypertarget{adding-a-legend}{%
\section{Adding a legend}\label{adding-a-legend}}

legend(``topright'', legend = c(``1 1'', ``1 0.5''), col = c(``red'',
``blue''), title = expression(paste(mu, '' ``, sigma)), title.adj =
0.75, lty = 1, lwd = 2)

\#pnorm \# Same sd, different mean \# Mean 0, sd 1 plot(x, pnorm(x, mean
= 0, sd = 1), type = ``l'', ylim = c(0, 1), ylab = ``F(x)'', lty=2, lwd
= 1, col = ``red'') \# Mean 3, sd 1 lines(x, pnorm(x, mean = 3, sd = 1),
col = ``blue'', lty = 1, lwd = 2)

\hypertarget{legend}{%
\section{Legend}\label{legend}}

legend(``topleft'', legend = c(``0 1'', ``3 1''), col = c(``red'',
``blue''), title = expression(paste(mu, '' ``, sigma)), lty = c(2,1),
lwd = c(1,2),)

\hypertarget{same-mean-different-sd}{%
\section{Same mean, different sd}\label{same-mean-different-sd}}

\hypertarget{mean-1-sd-1}{%
\section{Mean 1, sd 1}\label{mean-1-sd-1}}

plot(x, pnorm(x, mean = 1, sd = 1), type = ``l'', ylim = c(0, 1), ylab =
``F(x)'', lty=2, lwd = 1, col = ``red'') \# Mean 1, sd 0.5 lines(x,
pnorm(x, mean = 1, sd = 0.5), col = ``blue'', lty = 1, lwd = 2)

\hypertarget{legend-1}{%
\section{Legend}\label{legend-1}}

legend(``topleft'', legend = c(``1 1'', ``1 0.5''), col = c(``red'',
``blue''), title = expression(paste(mu, '' ``, sigma)), lty=c(2,1),
lwd=c(1,2) )

\#vi. normal distribution x = seq(-4, 8, 0.1)

\hypertarget{same-sd-different-mean}{%
\section{Same sd, different mean}\label{same-sd-different-mean}}

\hypertarget{mean-0-sd-1}{%
\section{Mean 0, sd 1}\label{mean-0-sd-1}}

plot(x, dnorm(x, mean = 0, sd = 1), type = ``l'', ylim = c(0, 0.6), ylab
= ``P(X==x)'', lty=1, lwd = 2, col = ``red'') \# Mean 3, sd 1 lines(x,
dnorm(x, mean = 3, sd = 1), col = ``blue'', lty = 2, lwd = 1)

\hypertarget{adding-a-legend-1}{%
\section{Adding a legend}\label{adding-a-legend-1}}

legend(``topright'', legend = c(``0 1'', ``3 1''), col = c(``red'',
``blue''), title = expression(paste(mu, '' ``, sigma)), lty=c(1,2),
lwd=c(2,1))

\hypertarget{same-mean-different-standard-deviation}{%
\section{Same mean, different standard
deviation}\label{same-mean-different-standard-deviation}}

\hypertarget{mean-1-sd-1-1}{%
\section{Mean 1, sd 1}\label{mean-1-sd-1-1}}

plot(x, dnorm(x, mean = 1, sd = 1), type = ``l'', ylim = c(0, 1), ylab =
``P(X==x)'', lty=1, lwd = 2, col = ``red'') \# Mean 1, sd 0.5 lines(x,
dnorm(x, mean = 1, sd = 0.5), col = ``blue'', lty = 2, lwd = 1)

\hypertarget{adding-a-legend-2}{%
\section{Adding a legend}\label{adding-a-legend-2}}

legend(``topright'', legend = c(``1 1'', ``1 0.5''), col = c(``red'',
``blue''), title = expression(paste(mu, '' ``, sigma)), lty = c(1,2),
lwd = c(2,1))

\#pnorm \# Same sd, different mean \# Mean 0, sd 1 plot(x, pnorm(x, mean
= 0, sd = 1), type = ``l'', ylim = c(0, 1), ylab = ``F(x)'', lty=1, lwd
= 2, col = ``red'') \# Mean 3, sd 1 lines(x, pnorm(x, mean = 3, sd = 1),
col = ``blue'', lty = 2, lwd = 1)

\hypertarget{legend-2}{%
\section{Legend}\label{legend-2}}

legend(``bottomright'', legend = c(``0 1'', ``3 1''), col = c(``red'',
``blue''), title = expression(paste(mu, '' ``, sigma)), lty = c(1,2),
lwd = c(2,1))

\hypertarget{same-mean-different-sd-1}{%
\section{Same mean, different sd}\label{same-mean-different-sd-1}}

\hypertarget{mean-1-sd-1-2}{%
\section{Mean 1, sd 1}\label{mean-1-sd-1-2}}

plot(x, pnorm(x, mean = 1, sd = 1), type = ``l'', ylim = c(0, 1), ylab =
``F(x)'', lty=1, lwd = 2, col = ``red'') \# Mean 1, sd 0.5 lines(x,
pnorm(x, mean = 1, sd = 0.5), col = ``blue'', lty = 2, lwd = 1)

\hypertarget{legend-3}{%
\section{Legend}\label{legend-3}}

legend(``bottomright'', legend = c(``1 1'', ``1 0.5''), col = c(``red'',
``blue''), title = expression(paste(mu, '' ``, sigma)), lty=c(1,2),
lwd=c(2,1))

\#vii. Log-normal distribution \#dlnorm curve(dlnorm(x, meanlog=0,
sdlog=.3), from=0, to=10, col=`blue') curve(dlnorm(x, meanlog=0,
sdlog=.5), from=0, to=10, col=`red', add=TRUE) curve(dlnorm(x,
meanlog=0, sdlog=1), from=0, to=10, col=`purple', add=TRUE)
legend(``topright'', title=``sdlog'', legend=c(``0.3'',``0.5'',``1''),
text.col=c(``blue'',``red'',``purple'') )

\#plnorm curve(plnorm(x, meanlog=0, sdlog=.3), from=0, to=10,
col=`blue') curve(plnorm(x, meanlog=0, sdlog=.5), from=0, to=10,
col=`red', add=TRUE) curve(plnorm(x, meanlog=0, sdlog=1), from=0, to=10,
col=`purple', add=TRUE) legend(``bottomright'', title=``sdlog'',
legend=c(``0.3'',``0.5'',``1''), text.col=c(``blue'',``red'',``purple'')
)

\#viii Weibull distribution \#dweibull curve(dweibull(x, shape=2,
scale=1), from=0, to=5, col=`black')

curve(dweibull(x, shape=3, scale=2), from=0, to=7, col=`red', add=TRUE)

curve(dweibull(x, shape=4, scale=3),\\
from=0, to=10, col=`blue', add=TRUE)

legend(``topright'', legend=c(``shape 2, scale 1'',``shape 3, scale
2'',``shape 4, scale 3''), text.col=c(``black'',``red'',``blue''),
cex=0.5 \#for setting the text size ) \#pweibull curve(pweibull(x,
shape=2, scale=1), from=0, to=5, col=`black')

curve(pweibull(x, shape=3, scale=2), from=0, to=7, col=`red', add=TRUE)

curve(pweibull(x, shape=4, scale=3),\\
from=0, to=10, col=`blue', add=TRUE) legend(``bottomright'',
legend=c(``shape 2, scale 1'',``shape 3, scale 2'',``shape 4, scale
3''), text.col=c(``black'',``red'',``blue''), cex=0.5 \#for setting the
text size )

\end{document}
